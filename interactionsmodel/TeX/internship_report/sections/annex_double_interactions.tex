\documentclass[../main.tex]{subfiles}

\begin{document}
\chapter{Double interactions equivalence}\label{chap:an_double_equiv}

%%%%%%%%%%%%%%%%%%%%%%%%%%%%%%%%%%%%%%%
%% FULL vs not FULL
%%%%%%%%%%%%%%%%%%%%%%%%%%%%%%%%%%%%%%%

For the interactions, the vector $\Theta$ considered was
in $\bbR^{\frac{p(p+1)}{2}}$.
This implies that we removed all the feature-interactions which occurred more
than once in the matrix $Z$.
However, we could also consider to keep all of the interactions,
meaning have $\widetilde Z\in\bbR^{n\times p^2}$ and
$\widetilde\Theta\in\bbR^{p^2}$
the full versions, with a supplementary constraint on the symmetry of the
coefficients in $\widetilde\Theta$.
With the notations of \Cref{chap:GLM}, if we exclude redundant interactions,
then $K=q=\nicefrac{p(p+1)}{2}$, else
$K=\widetilde q=p^2$. In the later case, we write $\widetilde Z$ the interaction
matrix. The problem is then:
\begin{equation}\label{eq:full}  \tag{$\cP_2$}
\widehat{p}
	\in \min_{\substack{\beta \in \bbR^p \\
	 				   \widetilde \Theta \in \bbR^{p\times p}}}
	\frac{1}{2n}\norm{y - X\beta - \widetilde Z\widetilde \Theta}_2^2 +
	g^{(1)}(\beta) + g^{(2)}(\widetilde \Theta) +
	\iota(\widetilde \Theta_{\tau_{\widetilde q}(i,j)}
	= \widetilde \Theta_{\tau_{\widetilde q}(j,i)}) \enspace.
\end{equation}
%
where $\iota(\widetilde \Theta_{\tau_{\widetilde q}(i,j)}=
	   \widetilde \Theta_{\tau_{\widetilde q}(j,i)}) = 0$
if the condition is verified and $+\infty$ o.w.
First, we study $y - X\beta - \widetilde Z\widetilde \Theta$, with
the symmetry constraint:
\begin{align*}
	y - X\beta - \widetilde Z\widetilde \Theta
	& = y - X\beta - \sum_{(i,j)\in [p]^2}
		\widetilde Z_{ij} \widetilde \Theta_{ij} \\
	& = y - X\beta - 2\sum_{i\in[p]}\sum_{j>i}
		\widetilde Z_{\tau_q(i,j)}
		\widetilde \Theta_{\tau_q(i,j)} -
			\sum_{i\in [p]} \widetilde Z_{\tau_q(i,i)}
			\widetilde \Theta_{\tau_q(i,i)} \\
	& = y - X\beta - \sum_{i\in[p]}\sum_{j>i}
		\widetilde Z_{\tau_q(i,j)} 2\widetilde \Theta_{\tau_q(i,j)} -
			\sum_{i\in [p]} Z_{\tau_q(i,i)} \Theta_{\tau_q(i,i)}
			\enspace.
\end{align*}

Posing $\Theta_{\tau_q(i,j)} = 2 \widetilde\Theta_{\tau_q(i,j)}$
for $i\neq j$, we thus get
\begin{align*}
	\frac{1}{2n}\norm{y - X\beta - \widetilde Z\widetilde \Theta}_2^2 &
	= \frac{1}{2n}\norm{y - X\beta -
		\sum_{i\in[p]}\sum_{j>i} Z_{\tau_q(i,j)} \Theta_{\tau_q(i,j)} -
		\sum_{i\in [p]} Z_{\tau_q(i,i)} \Theta_{\tau_q(i,i)}}_2^2 \\
	&= \frac{1}{2n}\norm{y - X\beta - Z \Theta}_2^2 \enspace.
\end{align*}

We execute the same decomposition on the penalties depending
on $\widetilde\Theta$.
This leads to modifying $g^{(2)}$ as follows:
\begin{align*}
	g^{(2)}(\widetilde\Theta)
	&= \lambda_{\widetilde\Theta, \ell_1}
	\sum_{i\in [p]}\abs{\widetilde\Theta_{\tau_q(i,i)}} +
		2\lambda_{\widetilde\Theta, \ell_1} \sum_{i>j}
			\abs{\widetilde\Theta_{\tau_q(i,j)}} +
 			\frac{\lambda_{\widetilde\Theta, \ell_2}}{2}
			\sum_{i\in [p]}\widetilde\Theta_{\tau_q(i,i)}^2 +
			\lambda_{\widetilde\Theta, \ell_2}\sum_{i>j}
			\widetilde\Theta_{\tau_q(i,j)}^2 \\
 	& = \lambda_{\Theta, \ell_1} \sum_{i\in [p]}
	 	\abs{\widetilde\Theta_{\tau_q(i,i)}} +
		\lambda_{\Theta, \ell_1} \sum_{i>j} \abs{\Theta_{\tau_q(i,j)}} +
	 	\frac{\lambda_{\Theta, \ell_2}}{2} \sum_{i\in [p]}\Theta_{\tau_q(i,i)}^2 +
	 	\frac{\lambda_{\Theta, \ell_2}}{2}\sum_{i>j}
	 	\left(
			\frac{\Theta_{\tau_q(i,j)}}{\sqrt{2}}
		\right)^2 \enspace,
\end{align*}
using the same variable change for non-diagonal terms. Considering as reference
the problem in dimension $q=\nicefrac{p(p+1)}{2}$, then we only need to multiply
instead of dividing in the $\ell_2$ penalty.
So to resume, solving $(\cP)$ in dimension $q$ is equivalent to solving:
\begin{equation}\label{eq:equiv_full}
	\begin{split}
	\widehat{p}
	  = \min_{\substack{\beta \in \bbR^p \\
	 					\widetilde \Theta \in \bbR^{p\times p}}}
	& \frac{1}{2n}\norm{y - X\beta - \widetilde Z\widetilde \Theta}_2^2 +
	g^{(1)}(\beta) \\
	& + \lambda_{\Theta, \ell_1}\norm{\widetilde\Theta}_1 +
	\frac{\lambda_{\Theta, \ell_2}}{2}\sum_{i\in[p]}
	\widetilde\Theta_{\tau_{\widetilde q(i,i)}}
	 + \lambda_{\Theta, \ell_2}\sum_{i\in [p]}\sum_{j>i}
	 \widetilde \Theta_{\tau_{\widetilde q(i,j)}}^2 +
	 \iota(\widetilde \Theta_{\tau_{\widetilde q}(i,j)}
	 =\widetilde \Theta_{\tau_{\widetilde q}(j,i)})
	 \enspace.
	\end{split}
\end{equation}

We then compute the proximal operator for the $\widetilde\Theta$ part with the
separability of the components \citep[p.~135]{Beck17}, which
leads to:
\begin{itemize}
	\item if $i=j$:
	\[\prox_{\mu\lambda_{\Theta, \ell_1}\left(|\cdot| +
	\frac{\lambda_{\Theta, \ell_2}/
		\lambda_{\Theta, \ell_1}}{2}(\cdot)^2\right)}(t)
		=\frac{\sign(t)}{1+\mu\lambda_{\Theta, \ell_2}}
		(|t| - \mu\lambda_{\Theta, \ell_1})_+ \enspace,\]
	\item if $i\neq j$:
	\[\prox_{\mu\lambda_{\Theta, \ell_1}\left(|\cdot| +
	2\frac{\lambda_{\Theta, \ell_2}}{
		\lambda_{\Theta,\ell_1}}(\cdot)^2\right)}(t)
		=\frac{\sign(t)}{1+2\mu\lambda_{\Theta, \ell_2}}
		(|t| - \mu\lambda_{\Theta, \ell_1})_+ \enspace.\]
\end{itemize}

\end{document}