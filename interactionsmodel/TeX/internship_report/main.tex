% !TEX options=--shell-escape
\documentclass[a4paper, twoside]{memoir}
\usepackage[english]{babel}
\usepackage{fullpage}
\usepackage{fancyhdr}
\usepackage{amsmath,amsfonts,amsthm,amssymb,nicefrac}

\usepackage{./sty/sty_tl}
\let\widering\relax
\usepackage[titlenumbered,ruled,noend]{algorithm2e}
\usepackage{hyperref}
\usepackage{cleveref}
\usepackage{multirow}
\usepackage{subcaption}
\usepackage{fancyvrb}
\usepackage[super]{nth}
\usepackage{tkz-fct}
\usepackage{enumitem}

\emergencystretch=1em

\graphicspath{{./prebuilt_images/}{./sources_images/}}
\renewcommand{\thesection}{\thechapter.\arabic{section}}

% include svg and get pdf on the fly
\newcommand{\executeiffilenewer}[3]{%
\ifnum\pdfstrcmp{\pdffilemoddate{#1}}%
{\pdffilemoddate{#2}}>0%
{\immediate\write18{#3}}\fi%
}
\newcommand{\includesvg}[1]{%
\executeiffilenewer{#1.svg}{#1.pdf}%
{inkscape -D #1.svg -o #1.pdf --export-latex}%
\input{#1.pdf_tex}%
}

% in algorithms
\SetKwFor{FOR}{For}{do}{end For}%
\hypersetup{
    colorlinks,
    linkcolor={red!50!black},
    citecolor={blue!50!black},
    urlcolor={blue!80!black}
    }

\usepackage[natbib=true,style=authoryear]{biblatex}
\setlength\bibitemsep{\baselineskip}
\bibliography{./sty/biblio.bib}

\setlength{\headsep}{13.6pt}
\setlength{\headheight}{13.6pt}

%%%%%%
% chapter style
%%%%%%
\usepackage{kpfonts}\setSingleSpace{1.1}\SingleSpacing
\usepackage{xcolor,calc}
\definecolor{chaptercolor}{gray}{0.8}%helper macros
\newcommand\numlifter[1]{\raisebox{-2cm}[0pt][0pt]{\smash{#1}}}
\newcommand\numindent{\kern37pt}
\newlength\chaptertitleboxheight
\makechapterstyle{hansen}{
    \renewcommand\printchaptername{\raggedleft}
    \renewcommand\printchapternum{%
    \begingroup%
    \leavevmode%
\chapnumfont%
\strut%
\numlifter{\thechapter}%
\numindent%
\endgroup%
}
\renewcommand*{\printchapternonum}{%
\vphantom{\begingroup%
\leavevmode%
\chapnumfont%
\numlifter{\vphantom{9}}%
\numindent%
\endgroup}
\afterchapternum}\setlength\midchapskip{0pt}
\setlength\beforechapskip{0.5\baselineskip}
\setlength{\afterchapskip}{3\baselineskip}
\renewcommand\chapnumfont{%
\fontsize{4cm}{0cm}%
\bfseries%
\sffamily%
\color{chaptercolor}%
}
\renewcommand\chaptitlefont{%
\normalfont%
\huge%
\bfseries%
\raggedleft%
}%
\settototalheight\chaptertitleboxheight{%
\parbox{\textwidth}{\chaptitlefont \strut bg\\bg\strut}}
\renewcommand\printchaptertitle[1]{%
\parbox[t][\chaptertitleboxheight][t]{\textwidth}{%
\chaptitlefont\strut ##1\strut}%
}
}
\chapterstyle{hansen}
\aliaspagestyle{chapter}{empty} % just to save some space

\setlength{\headwidth}{\textwidth}
\pagestyle{companion}
\definecolor{darkgreen}{rgb}{0.09, 0.45, 0.27}

\setsecnumdepth{subsection}
\settocdepth{subsection}
\Crefname{algocf}{Algorithm}{Algorithms}

\usepackage{subfiles}
\usepackage{minted}
\usepackage{csquotes}


\begin{document}
\begin{titlingpage}
    \begin{figure}
        \begin{subfigure}{.45\paperwidth}
    \includegraphics[scale=1]{Logo}
        \end{subfigure}
    \hfill
        \begin{subfigure}{.45\paperwidth}
    \includegraphics[scale=.15]{imag_logo}
        \end{subfigure}
    \end{figure}
        \vspace{1cm}
    \begin{center}
        Internship report \vspace{.5cm} Specialty: \\ {\scshape Biostatistics}
    \end{center} \vspace{2cm}
     \huge {\bfseries \begin{center}
        High dimensional optimization for penalized
        linear models with first order interactions using graphics card
        computational power
    \end{center} \hrule }
    \begin{flushright} Tanguy Lefort \end{flushright}
    \vspace{1cm}
    \normalsize
    \textbf{Supervised by:}
      \begin{center}
         \begin{tabular}{ll  }
              M. Joseph Salmon & University of Montpellier\\
              M. Benjamin Charlier & University of Montpellier\\
         \end{tabular}
        \end{center}
    \normalsize\vspace{2cm} \textbf{Presented Thursday, July 20th 2021 with jury}:
    \begin{center}
        \begin{tabular}{ll}
            M. Jean-Michel Marin & University of Montpellier \\
            M. Paul Bastide & University of Montpellier \\
            Mme. Elodie Brunel-Piccinini & University of Montpellier \\
        \end{tabular}
    \end{center}
\end{titlingpage}

\setcounter{page}{0}
\pagenumbering{roman}

\thispagestyle{plain}
\addcontentsline{toc}{chapter}{Remerciements - Abstract}
\begin{center}
    \textbf{\Huge{Remerciements}}
\end{center}

Je tiens tout d'abord à remercier mes encadrants Joseph Salmon et Benjamin
Charlier pour leur supervision tout au long de ce stage.
Des parties plus théoriques à la programmation numérique, leurs conseils et
explications auront été très utiles et le seront tout autant pour la suite.

Je remercie également Florent Bascou pour m'avoir introduit à son travail de thèse.
Nos nombreuses discussions ont permis de comprendre rapidement certaines subtilités
du sujet mais aussi de découvrir d'autres méthodes possibles.

Je souhaite aussi remercier Sophie Lèbre et Thomas Moreau.
Sophie pour ses explications par rapport à la partie génomique
de ce rapport.
Cela a permis d'amener un apport pratique à ce travail.
Thomas pour m'avoir guidé dans mes contributions à la librairie \texttt{BenchOpt}
et la confiance accordée pour apporter des modifications au site de la librairie.

Enfin merci à Jean-Michel Marin pour le temps consacré à ce rapport.

\medskip
\begin{center}
    \textbf{\Huge{Abstract}}
\end{center}
\paragraph*{French:}

Le modèle linéaire est utilisé en statistiques pour sa simplicité,
et l'interprétabilité des résultats obtenus.
Sur des données génomiques, les dimensions très grandes imposent d'utiliser des
méthodes robustes qui sélectionnent les variables actives pour avoir des résultats
interprétables pour des biologistes.
En plus de l'effet de nos variables on cherche aussi à capturer les effets des
interactions, ce qui augmente encore la dimension du problème et les phénomènes
de colinéarité.
Pour pallier à cela, nous considérons l'Elastic-Net sur le problème augmenté.
La descente par coordonées \citep{wu2008coordinate} est très utilisée
pour résoudre ce type de problèmes, mais ce n'est pas l'unique possibilité.
Nous utilisons la structure du problème avec interactions du premier ordre pour
paralléliser des algorithmes de descente de gradient proximal.
Ceux-ci sont connus pour être plus lents à converger du point de vue de la
complexité, mais utiliser la parallélisation sur carte graphique permet dans
certaines situations d'être plus rapide.
Ce travail sur les méthodes d'optimisation s'inscrit dans le
développement de la librairie \texttt{BenchOpt} permettant de comparer
facilement différents algorithmes.

\paragraph*{English:}

Linear models are used in statistics for their simplicity and the
interpretability of the results.
On genomics datasets, large dimensions need robust methods that induce sparsity
to select interpretable active features for biologists.
In addition to the main features, we also capture the effects of the interactions,
which increase the dimension of the problem and the multicolinearity.
To counteract these issues, we use the Elastic-Net on the augmented problem.
Coordinate Descent \citep{wu2008coordinate} is mostly used nowadays for that,
but there are other methods available.
We exploit the structure of our problem with first order interactions to
use parallelized proximal gradient descent algorithms.
Those are known to be more computationally demanding in order of magnitude,
but parallelizing on a graphics card let us be faster in some situations.
This work is set in the development of the \texttt{BenchOpt} library.
This library let us easily compare different optimization algorithms.

\paragraph{Github repository}
The code to generate the Figures and other results in this report is available at:
\begin{center}
    \url{https://github.com/tanglef/interactionsmodel}
\end{center}

\newpage
\begin{KeepFromToc}
\tableofcontents
\end{KeepFromToc}

\newpage
\pagenumbering{arabic}
\chapter{Introduction}
\subfile{sections/introduction.tex}


\chapter{Elastic-Net estimator with interactions}\label{chap:GLM}
\subfile{sections/glm_interactions.tex}


\chapter{Parallel numerical scheme for linear models with interactions}
\label{chap:an}
\subfile{sections/ana_num.tex}


\chapter{Application to genomics dataset} \label{chap:genom}
\subfile{sections/condition_numbers.tex}


\chapter{The BenchOpt library}\label{chap:benchopt}
\subfile{sections/benchopt.tex}

\chapter{Conclusion}
\subfile{sections/conclusion.tex}

\printbibliography

\appendix
\subfile{sections/annex_cv.tex}
\subfile{sections/annex_double_interactions.tex}

\end{document}