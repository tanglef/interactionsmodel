\documentclass[10pt,aspectratio=43]{beamer}

\usepackage{graphicx}
\graphicspath{{../internship_report/prebuilt_images/}}
\usepackage[rotationcw % clockwise, default is counterclockwise
			]{beamerthemeGlobalUniNA}

\usepackage[utf8]{inputenc}
\usepackage[english]{babel}
\usepackage[T1]{fontenc}
\usepackage{csquotes}
\usepackage{amsmath,amsfonts,amsthm,amssymb}
\usepackage{copyrightbox}
\usepackage{cleveref}
\usepackage{hyperref}
\usepackage[scaled]{beramono}
\usepackage[scale=1.05]{AlegreyaSans}
\usepackage{sty_tl}
\usepackage{subcaption}
\usepackage{minted}
\pgfplotsset{compat=1.17}
\usetikzlibrary{spy}
\usepackage[round]{natbib}
\bibliographystyle{plainnat}

%%%%%%%%%%%%%%%%%%%%%%%%%%%%%%%%%%%%%%%%%%%%%%%%%%%%%%%%%%%%%%%%%%%%%%%%%%%%%%%
% footnote setting: (1), (2), etc.
\usepackage{fnpct}

% Configure style for custom doubled line
\newcommand*{\doublerule}{\hrule width \hsize height 1pt \kern 0.5mm \hrule width \hsize height 2pt}

% Configure function to fill line with doubled line
\newcommand\doublerulefill{\leavevmode\leaders\vbox{\hrule width .1pt\kern1pt\hrule}\hfill\kern0pt }


\newcommand{\mytheorem}[2]{
    \doublerulefill\ \framebox{\textbf{#1}}\ \doublerulefill
    \vspace{0.1cm}
    #2
    \doublerulefill
}

\definecolor{javared}{rgb}{0.6,0,0} % for strings
\definecolor{javagreen}{rgb}{0.25,0.5,0.35} % comments
\definecolor{javapurple}{rgb}{0.5,0,0.35} % keywords
\definecolor{javadocblue}{rgb}{0.25,0.35,0.75} % javadoc
\definecolor{marron}{rgb}{0.64,0.16,0.16}
\definecolor{orange_js}{RGB}{230,159,0}

\newcommand{\mybold}[1]{\textcolor{marron}{\textbf{#1}}}
\newcommand{\cRm}[1]{\textsc{\romannumeral #1}}

\usepackage{csquotes}

%%%%%%%%%%%%%%%%%%%%%%%%%%%%%%%%%%%%%%%%%%%%%%%%%%%%%%%%%%%%%%%%%%%%%%%%%%%%%%%
%%%%%%%%%%%%%%%%%%%%%%%%%%%%%%%%%%%%%%%%%%%%%%%%%%%%%%%%%%%%%%%%%%%%%%%%%%%%%%%
% HEADER
%%%%%%%%%%%%%%%%%%%%%%%%%%%%%%%%%%%%%%%%%%%%%%%%%%%%%%%%%%%%%%%%%%%%%%%%%%%%%%%
%%%%%%%%%%%%%%%%%%%%%%%%%%%%%%%%%%%%%%%%%%%%%%%%%%%%%%%%%%%%%%%%%%%%%%%%%%%%%%%

\title[] %shown at the top of frames
{High dimensional penalized linear models with interactions using graphics card
} %shown in title frame
\subtitle{Internship supervised by Joseph Salmon and Benjamin Charlier}

\date{07-2021} % explicitly set date instead of \today

\author[]%shown at the top of frames
{%shown in title frame
	{Lefort Tanguy}%
}

\institute[
]
{% is placed on the bottom of the title page
    University of Montpellier
}

\titlegraphic{% logos are put at the bottom-right part of the page
    \includegraphics[width=3cm]{Logo}\hspace{1cm} % also support multi-logos
    \includegraphics[width=3cm]{imag_logo}~ % up to 3 (after it gets messy)
    %\includegraphics[width=2cm]{Logo.pdf} % if more, combine them in one image.
}

\setbeamercolor{itemize subitem}{fg=javadocblue}
\setbeamertemplate{itemize subitem}[triangle]

%%%%%%%%%%%%%%%%%%%%%%%%%%%%%%%%%%%%%%%%%%%%%%%%%%%%%%%%%%%%%%%%%%%%%%%%%%%%%%%
%%%%%%%%%%%%%%%%%%%%%%%%       PLAN      %%%%%%%%%%%%%%%%%%%%%%%%%%%%%%%%%%%%%%
%%%%%%%%%%%%%%%%%%%%%%%%%%%%%%%%%%%%%%%%%%%%%%%%%%%%%%%%%%%%%%%%%%%%%%%%%%%%%%%

\begin{document}
\maketitle


%%%%%%%%%%%%%%%%%%%%%%%%%%%
% Table of contents
%%%%%%%%%%%%%%%%%%%%%%%%%%%
\begin{frame}{Content}{}
    \tableofcontents
\end{frame}

\section*{Introduction}

%%%%%%%%%%%%%%%%%%%%%%
% Introduction
%%%%%%%%%%%%%%%%%%%%%%
\begin{frame}{Introduction}{The Linear Model}
We denote $X \in \bbR^{n \times p}, y \in \bbR^n$ and
$\beta \in \bbR^p$ such that $y\simeq X\beta$. \\
Ordinary least squares:
\[\hat\beta^{ls} = \argmin_{\beta} \frac{1}{2n}\|y-X\beta\|_2^2
\Longleftrightarrow \hat \beta = (X^\top X)^{-1}X^\top y \enspace.\]

Problems with high dimensions:
\begin{itemize}
\item if $p>n$ we lose the uniqueness,
\begin{onlyenv}<2>
    \begin{itemize}
        \item  \color{red}{make the problem strictly convex.}
    \end{itemize}
\end{onlyenv}
\item $X^\top X$ may be ill conditioned
($\kappa = \frac{\text{largest singular value}}{
    \text{smallest singular value}} \gg $ )
    because of multicolinearity amongst features,
\begin{onlyenv}<2->
    \begin{itemize}
        \item \color{red}{Shift spectrum by a small quantity using
        $\ell_2$ penalty.}
    \end{itemize}
\end{onlyenv}
\item too many active features is not interpretable (genomics dataset),
\begin{onlyenv}<2->
    \begin{itemize}
        \item \color{red}{Feature selection using $\ell_1$ penalty.}
    \end{itemize}
\end{onlyenv}\end{itemize}
\end{frame}

\begin{frame}{Introduction}{Elastic-Net\footnote[frame]{\citet{Zou_Hastie05}}
             estimator}
Combination of LASSO\footnote[frame]{\citet{Tibshirani96}} and
Ridge\footnote[frame]{\citet{Tikhonov43}}:
\begin{block}{Elastic-Net}
Considering $\lambda_1,\lambda_2>0$ penalties,
\begin{align*}
    \hat\beta^{enet} \in \frac{1}{2n}\norm{y - X\beta}_2^2
    + \lambda_1 \norm{\beta}_1
    + \frac{\lambda_2}{2} \norm{\beta}_2^2 \enspace.
\end{align*}
\end{block}
\begin{itemize}
    \item The higher $\lambda_1$, the sparser $\beta$.
    \item The higher $\lambda_2$, the closer $\beta$ will be to $0_p$.
\end{itemize}
\end{frame}




\begin{frame}{Conclusion}
    dfghhgfdsdfgfd
\end{frame}


%%%%%%%%%%%%%%%%%
% Biblio
%%%%%%%%%%%%%%%%%

\begin{frame}[allowframebreaks]{}
    \frametitle{References}
    \bibliography{../internship_report/sty/biblio.bib}
\end{frame}

\end{document}
