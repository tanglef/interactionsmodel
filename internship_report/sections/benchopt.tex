\documentclass[../main.tex]{subfiles}

\begin{document}

Throughout this report, multiple benchmarks were executed, with different
datasets, solvers and parameters. As a general rule, scientific results should
be reproducible, but in practice, this can become quite tideous and very time
consuming. This is where \texttt{BenchOpt} can help.
\begin{center}
    \url{https://benchopt.github.io/}
\end{center}
%
\paragraph*{}
%
Created by Thomas Moreau\footnote{\url{https://tommoral.github.io/about.html}},
Alexandre Gramfort\footnote{\url{http://alexandre.gramfort.net/}},
Joseph Salmon\footnote{\url{http://josephsalmon.eu/}},
Tom Dupré la Tour\footnote{\url{https://tomdlt.github.io/}}
and Mathurin Massias\footnote{\url{https://mathurinm.github.io/}},
this \texttt{Python} library enables the users to easily compare cross-languages
optimization solvers and with easy reproducibility.
In practice, we already saw an example with \Cref{fig:benchopt} of what this
library can produce.
And how it works is quite simple if we look at what is an optimization problem.
Indeed, it is made of (at least) three components:
\begin{enumerate}[label=$\bullet$]
    \item an objective to minimize,
    \item a solver that will run until a criterion stops it (the criterion can
    be a maximum number of epochs or a precision for example),
    \item a dataset which can be simulated or downloaded.
\end{enumerate}
Each of these components can be matched with a single file.
And to make this even easier, templates for each are available on the
website and repository.
\begin{figure}[h!]
    \centering
    \includegraphics[width=.7\textwidth]{benchopt_res_index}
    \caption{Index page of the \texttt{BenchOpt} results webpage.
    Each problem can be clicked on and lead to the available files.
    Hovering on one displays the number of files inside.}
    \label{fig:index_results}
\end{figure}
Several optimization problems like the LASSO, Logistic Regression with penalties,
Ordinary Least Squares \dots are already available on the main repository of the
library.
And after cloning the chosen one (for example \texttt{benchmark\_lasso}), a simple

\begin{center}
\begin{minipage}{8cm}
\begin{minted}[
    frame=lines,
    framesep=2mm,
    baselinestretch=1.2,
    bgcolor=chaptercolor]{shell-session}
    $ benchopt run ./benchmark_lasso
\end{minted}
\end{minipage}
\end{center}
will run all solvers on the datasets. Option flags are available to only run on
some.
But sometimes one might want to see how behaves a solver on some examples
before using it.
And this, without needing to code it and take a certain time to
run the comparisons.
This is where the other side of the \texttt{BenchOpt} website comes in.
This website had a pre-existing base that we worked on during the internship to
present more informations to the users.
\begin{center}
    \url{https://benchopt.github.io/results/}
\end{center}
%
\paragraph*{Choose the problem to solve.}
%
The \texttt{results} part's strength of \texttt{BenchOpt} results in the easy
access and visualizations of benchmarks quickly, on a website, but with all the
informations that one might need.
Currently there are $8$ different optimization problems available
(see \Cref{fig:index_results}) and several benchmarks in each.
%
\paragraph*{Choose the benchmark}
%
As aforementioned, on one problem we can benchmark several solvers on
several datasets.
Sometimes, the hardware plays an important role in the resulting
performances.
For example, with CPUs, an Intel Core i3 can not be expected
to be as fast as an Intel Core i9.
So for the benchmarks to be useful, hardware
related informations are displayed (\Cref{fig:benchs}), with more informations
one click away.
As more users come, more examples are needed for the community. With a single
command\footnote{More information at \url{https://benchopt.github.io/publish.html\#publish-doc}}
users can publish their results on the website.
This participative building can lead to many available files.
This is why we created a filter for the main system
information in the sidebar (another filter is available above the table where user
can directly write words).
\begin{figure}[h]
    \centering
    \includegraphics[width=\textwidth]{benchopt_filter}
    \caption{Page of the availables benchmark for the Logistic Regression with
    $\ell_2$ penalty. There are three benchmarks available. The last one was
    run on four different datasets. Only the last two have system information
    available. Filters are inside the sidebar on the left.}
    \label{fig:benchs}
\end{figure}
Users only have to click on the benchmark they want, select the visualization and
get an interactive plot using \texttt{Plotly} directly available.
%
\paragraph*{Programming side}
%
The goal of this static and github-pages supported website is that it can expand easily
and with as few maintenance as possible.
Typically, if another problem is added, a new \emph{box} does not need
to be manually added to the main page.
This is made possible by \texttt{Mako}\footnote{\url{https://www.makotemplates.org/}}.
This library lets us use some \texttt{Python} programming directly inside an
\texttt{HTML} template.
The syntax is very close to \texttt{PHP}, only without the need of a server.
\texttt{CSS} is loaded to modify the style of the objects (typically the colors,
sizes and some animations).
And finally \texttt{Javascript} performs actions like on-click responses when
the user clicks on a button (like with the filters).
%
\paragraph*{Portability}
%
Hovering, appearing sidebar, buttons and on-click actions are often well working
on computers.
However, mobile devices are a non-negligible part of devices from
which we browse the internet everyday.
So we adapted the website to mobile devices, for example by removing the sidebar
and using a filter menu (\Cref{fig:mobile_filter}) to have a better experience.
\begin{figure}[h]
    \centering
    \includegraphics[width=.4\textwidth]{mobile_filter.pdf}
    \caption{Mobile filtering is not made with the sidebar, but in a menu over
    the page. This allows flexibility for different smaller screen sizes.}
    \label{fig:mobile_filter}
\end{figure}
%
\paragraph*{Results page}
%
After selecting the benchmark of your choice, the page of the results appear.
In the system informations, we have besides the number of CPUs, ram, platform
and processor that were available before.
Added to them are the \texttt{Numpy} and \texttt{Scipy} versions alongside
the \texttt{Blas} and \texttt{Lapack} libraries.
Several dropdown menus allow us to change the datasets, objectives measured and
also the kind of the plot (suboptimality curve, histogram,\dots).
One last button was added using \texttt{PlotlyJs} to be able to toggle
between a plot in \emph{loglog} scale and \emph{semilog-y}.
\begin{figure}[h]
    \centering
    \includegraphics[width=\textwidth]{benchopt_results}
    \caption{Page of the results for the quantile regression.}
    \label{fig:benchopt_quantile_reg}
\end{figure}
\end{document}